\documentclass{article}
\usepackage[letterpaper,margin=3cm]{geometry} 
\usepackage{graphicx} % Required for inserting images
\usepackage[spanish]{babel}
\usepackage[usenames]{color}
\usepackage{hyperref}
\hypersetup{colorlinks=true, linkcolor = black, citecolor= black}
\usepackage{booktabs}
\usepackage{natbib}
\usepackage{tikz}
\usepackage{float} % para usar la opción [H]
\bibliographystyle{agsm} 
\usepackage{diagbox} % Para la línea diagonal
\usepackage{listings}
\usepackage{xcolor} % Paquete para definir y usar colores
\usepackage{parskip}
\usepackage{fancyhdr}
\usepackage{amsmath}
\usepackage{titlesec}
\usepackage{lipsum}  % Solo para texto de relleno

% Configuración de fancyhdr
\pagestyle{fancy} % Usa el estilo fancyhdr
\fancyhf{} % Borra todos los encabezados y pies de página

\renewcommand{\headrulewidth}{0pt}
\renewcommand{\footrulewidth}{0pt} % Desactiva la línea horizontal predeterminada en el pie

\fancyhead[L]{\raisebox{0.20cm}{\textbf{Métodos Computacionales en Obras Civiles}}}
\fancyhead[R]{\raisebox{0.1cm}{\includegraphics[width=0.25\linewidth]{logo principal.jpg}}}
\fancyhead[C]{\rule{\textwidth}{0.6pt}}
\fancyfoot[C]{\rule{\textwidth}{0.6pt}}
\fancyfoot[R]{\raisebox{-1.5\baselineskip}{\thepage}}

% Ajustes de geometría
% Ajustes de geometría
\geometry{
  top=3.5cm, % Aumenta el espacio en la parte superior para subir el encabezado
  bottom=2.5cm,
  headheight=2.5cm % Aumenta la altura del encabezado si es necesario
}

% Redefinir comando part
\titleclass{\part}{top} % Make part like a class
\titleformat{\part}[display]
  {\normalfont\huge\bfseries\centering}{\thepart}{20pt}{\Huge}
\titlespacing*{\part}{172.5pt}{-60pt}{10pt}
\titleformat{\part}
  {\normalfont\huge\bfseries}{}{0pt}{}

% Asegúrate de usar esto para mantener el estilo en las páginas de las partes
\titleformat{\part}[display]
  {\normalfont\huge\bfseries}{}{0pt}{}
  [\thispagestyle{fancy}] % Aplica el estilo fancy a las páginas de las partes


% Definición de colores al estilo Visual Studio Code
\definecolor{codegreen}{rgb}{0.25,0.49,0.48} % Comentarios
\definecolor{codegray}{rgb}{0.5,0.5,0.5} % Números y anotaciones
\definecolor{codepurple}{rgb}{0.58,0,0.82} % Palabras clave
\definecolor{backcolour}{rgb}{0.95,0.95,0.92} % Color de fondo

% Configuración del estilo de las celdas de código
\lstset{
    backgroundcolor=\color{backcolour},   % color de fondo; necesita que el paquete color o xcolor esté cargado
    commentstyle=\color{codegreen},       % estilo de comentarios
    keywordstyle=\color{codepurple},      % estilo de palabras clave
    numberstyle=\tiny\color{codegray},    % estilo de los números de línea
    stringstyle=\color{red},              % estilo de las cadenas de texto
    basicstyle=\ttfamily\small,           % estilo del texto básico
    breakatwhitespace=false,              % ajustes de líneas sólo en espacios en blanco
    breaklines=true,                      % ajustar las líneas si son muy largas
    captionpos=b,                         % posición de la leyenda (abajo)
    keepspaces=true,                      % preserva los espacios en el texto; útil si se usa monoespaciado
    numbers=left,                         % dónde poner los números de línea
    numbersep=5pt,                        % qué tan lejos están los números de línea del código
    showspaces=false,                     % mostrar espacios con subrayados particulares; reemplaza 'showstringspaces'
    showstringspaces=false,               % subrayar los espacios dentro de las cadenas solo
    showtabs=false,                       % mostrar tabulaciones en el código con subrayados particulares
    tabsize=2,                            % tamaños de tabulación a 2 espacios
    language=TeX,                         % lenguaje del código
    morecomment=[l]\#,                    % reconocer # como inicio de comentario en Python
    frame=single,                         % agregar un marco simple alrededor del código
    rulecolor=\color{black}               % color del marco
}



\begin{document}
%----------------------------------------------------------------------------------------
% PORTADA
%----------------------------------------------------------------------------------------
\begin{titlepage}%Inicio de la carátula, solo modificar los datos necesarios
\newcommand{\HRule}{\rule{\linewidth}{0.5mm}} 
\center 
%----------------------------------------------------------------------------------------
%	ENCABEZADO
%----------------------------------------------------------------------------------------
\includegraphics[width=10cm]{Logo principal.jpg}\\ % Si esta plantilla se copio correctamente, va a llevar la imagen del logo de la facultad.OBS: Es necesario incluir el paquete: graphicx
\vspace{3cm}
%----------------------------------------------------------------------------------------
%	SECCION DEL TITULO
%----------------------------------------------------------------------------------------
\HRule \\[0.4cm]
{ \huge \bfseries Entrega 0}\\[0.4cm] % Titulo del documento
{ \huge \bfseries Metodos Computacionales en OOCC, IOC 4201}\\[0.4cm] % Titulo del documento
\HRule \\[1.5cm]
 \vspace{5cm}
%----------------------------------------------------------------------------------------
%	SECCION DEL AUTOR
%----------------------------------------------------------------------------------------
\begin{flushright}
    { \textbf{Profesor:}
    Patricio Moreno\\
    \vspace{0.2cm}
    \textbf{Ayudante:}
    Maximiliano Biasi\\
    \vspace{0.2cm}
    \textbf{Alumno:}
    Bernardo Caprile Canala-Echevarría\\
}
\end{flushright}
\vspace{1cm}
%----------------------------------------------------------------------------------------
%	SECCION DE LA FECHA
%----------------------------------------------------------------------------------------
{\large \textbf{\today}}\\[2cm] % El comando \today coloca la fecha del dia, y esto se actualiza con cada compilacion, en caso de querer tener una fecha estatica, reemplazar el \today por la fecha deseada
\end{titlepage}
%----------------------------------------------------------------------------------------
%  INDICE
%----------------------------------------------------------------------------------------
\newpage
\tableofcontents
\newpage

%----------------------------------------------------------------------------------------
\part{Entrega 0}
\section{Introducción}
Para obras en las que se debe construir a nivel subacuático o con un nivel freático alto, es necesario el uso de ataguías. Estas estructuras temporales permiten construir de forma segura y eficiente en condiciones de humedad. Es importante, antes de instalar las ataguías, tener conocimiento de la profundidad a la que se van a hundir, la presión que se va a contener, tanto del agua como de otros factores, y la cantidad de agua que se va a bombear. De lo contrario, se pondría en riesgo la vida de los trabajadores y la maquinaria. Por ello, en esta entrega se presentarán esquemas de redes de flujo, caudales de infiltración, presiones de poros, gradientes hidráulicos, entre otros, de tres casos distintos de ataguías de tablaestaca.

\newpage

\section{Resultados}
\subsection{Redes de flujo}
A continuación, se muestran los esquemas de las redes de flujo de las 3 ataguías. 

\begin{figure}[h]
    \centering
    \begin{minipage}{0.32\textwidth}
        \centering
        \includegraphics[width=\textwidth]{graficos/At_caso1.png}
        \caption{Ataguía con el caso 1}
        \label{fig:At_caso1}
    \end{minipage}
    \hfill
    \begin{minipage}{0.32\textwidth}
        \centering
        \includegraphics[width=\textwidth]{graficos/At_caso2.png}
        \caption{Ataguía con el caso 2}
        \label{fig:At_caso2}
    \end{minipage}
    \hfill
    \begin{minipage}{0.32\textwidth}
        \centering
        \includegraphics[width=\textwidth]{graficos/At_caso3.png}
        \caption{Ataguía con el caso 3}
        \label{fig:At_caso3}
    \end{minipage}
\end{figure}

Como se puede apreciar, en la figura \ref{fig:At_caso1} la tablaestaca no está enterrada, mientras que la tablaestaca de la figura \ref{fig:At_caso2} está enterrada a una profundidad de 2.4 metros. Por último, en la figura \ref{fig:At_caso3} la tablaestaca está enterrada a una profundidad de 5.8 metros.

\end{document}